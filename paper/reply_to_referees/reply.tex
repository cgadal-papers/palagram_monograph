\documentclass[12pt]{article}
\usepackage{a4wide}
\usepackage[usenames]{color}

\usepackage{graphics}
\usepackage{epsfig}
\usepackage{amssymb}
\usepackage{subfigure}
\usepackage{pifont}
\usepackage{float}
\usepackage{natbib}
\usepackage[top=2cm, bottom=2cm, left=2cm, right=2cm]{geometry}

\usepackage[dvipsnames]{xcolor}
\newcommand{\julien}{\textcolor{WildStrawberry}}
\newcommand{\laurent}{\textcolor{ForestGreen}}
\newcommand{\tocheck}{\textcolor{Mulberry}}
\newcommand{\marie}{\textcolor{red}}

\newcommand{\MinorReply}[3]{\filbreak\noindent
\textbf{Referee:}
\emph{#1 -- #2}

\textbf{Reply:} {\color{blue}{#3}}
}

\newcommand{\MajorReply}[2]{\noindent
\textbf{Referee:}
\emph{#1}

\medskip
\noindent\textbf{Reply:} {\color{blue}#2}
\vspace*{0.4 cm}}


\begin{document}


\noindent Dear Editor,\\

\noindent We thank you very much for considering our chapter ``Chapter 16: Particle-laden gravity currents: the lock-release slumping regime at the laboratory scale", by C. Gadal \emph{et al}, for publication in a book for the \emph{Geophysical Monograph Series}. We are grateful for the thoughtful referee reports and for their comments and suggestions. \\

\noindent
Please find below a detailed reply to each of the comments and questions by the referees. Additionally, places where the text has been correspondingly edited, appear in red (removed text) and blue (added text) in the file \emph{diff.pdf}. \\

\noindent Sincerely, \\

\noindent The authors


%__________________________________________________________________________
\newpage
\section{Reply to referee 1}

\MajorReply{The authors discuss gravity currents in the lock-release configuration, where the densest phase is represented by sediments subject to a progressive decrease in concentration during propagation. The work is supported by an extensive data set of three experimental configurations with varying bottom slope. Two of these configurations are simulated with an open source code to which a module, "SedFoam", published by some of the co-authors of this paper, has been added. This is followed by a description of the experiments and an interpretation of the results, categorising the effects of some of the many parameters controlling the physical process. The paper is of particular interest, even though it does not provide clear results for some of the investigations carried out.
}{}

\begin{itemize}
    \item \MinorReply{Introduction}{The literature, although rich, lacks a number of contributions of some importance, mainly because they are themselves experimental. See, for example, the work of Zemach et al. 2017, which deals with some more complex configurations, with non-rectangular geometry, with stratified fluid, possibly in the partial depth lock release configuration. The various contributions should be catalogued and described in an appropriate way.}{The introduction has been rewritten according to various comments of the referees and other readers. Some references have been added. Note that we can not provide an extensive catalog of the complete literature related to gravity currents in such an introduction, bound to have a manuscript of reasonable length. We have then focused on the literature that shares the most scientific objectives with the present contribution.}

    \item \MinorReply{Figure 1}{the q-v panels look out of place and the sequence is horizontal while the experiments are vertical. It is recommended that they be placed in a separate figure.}{There are already many figures compared to the text length, and we do not wish to emphasize the numerical simulations by adding a dedicated figure. Hence, we modified figure 1 to better separate the q-v panels from the rest.}

    \item \MinorReply{Figure 3a}{Figure 3a is not meaningful due to the large amount of data shown. It is advisable to scale it down in dimensionless form, as is done in Figure 3b for 'selected runs', which should be specified.}{Done.}

    \item \MinorReply{Figure 3b}{the alpha symbol is flanked by an arrow which appears to be meaningless. I assume that the arrow should be reversed.}{The flanking arrows have been removed, and the figure caption has been modified accordingly.}

    \item \MinorReply{References}{the book by Ungarish should be better updated to the last edition.}{Done.}
\end{itemize}


%__________________________________________________________________________
\newpage
\section{Reply to referee 2}

\MajorReply{This is generally a good contribution with a lot of interesting results. The following are few comments the authors might want to make use.}{}

\begin{itemize}
    \item \MinorReply{Line 64}{Line 64 is unclear}{Done.}
    
    \item \MinorReply{Line 78}{The sentence starting in line 78 is too convoluted. Needs restructuring.}{Done.}

    \item \MinorReply{Line 102}{what does it mean by $h_c$ can have different scaling?}{This sentence has been removed, and this part of the text re-written for the sake of clarity.}

    \item \MinorReply{Lines 104--107}{It is unclear what the authors mean gt detachment. Also “similarly” may need to change to “similar”}{This part of the text has been re-written for the sake of clarity.}

    \item \MinorReply{Section 1.1}{gives several different expressions for $X_t$. Some clarification as to where each apply will be useful.}{The physical process associated with each expression of $X_{t}$ is stated along the section when $X_{t}$ is defined. Now, the first transition to occur will depend on the prevailing mechanism, and as such on where the experimental run is in the parameter space. This is now clarified in the introduction.}

    \item \MinorReply{Lines 133-134}{where do length and time scale of particles play a role?}{This sentence has been modified by adding examples.}

    \item \MinorReply{Line 211-212}{what does it mean 2D vertical and turbulence averaged? Are these 2D RANS simulations? Are the 2 coordinates horizontal, since vertical is averaged?}{\julien{julien}}

    \item \MinorReply{In figure 2}{the color of the symbols outline must be the same. Otherwise it is hard to relate to the color legend}{The color of the outline of the symbols is used to show the difference between two different kinds of runs, as explained in the caption. We prefer to leave it this way.}

    \item \MinorReply{figure 2}{some of the volume fraction data goes to 1.}{\marie{marie}}

    \item \MinorReply{}{Curve fits with such a large data scatter, requires some justification}{Here, we do not fit a proper analytical model or check the validity of its prediction against the data. We merely aim to show that a simple dimensional model incorporating the required mechanisms is able to reproduce the observed variation of the data while keeping the scaling constants within reasonable values. In addition, we now clarify that the only true adjustable parameters are $S$ for the slope effect and $Re_{c}$ for the effect of the volume fraction, as $C_{\rm d}$ and $Fr_{0}$ are strongly bound by values in the literature.}

    \item \MinorReply{Line 461}{even even}{Done.}

    \item \MinorReply{Line 466}{This should be an important….}{Done.}
\end{itemize}

\end{document}